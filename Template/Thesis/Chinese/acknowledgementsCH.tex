\begin{acknowledgementsCH}
 這裡將簡單介紹如何利用\LaTeX\ 來編輯你的畢業論文,若不知道\LaTeX\ 是什麼或是沒有概念的話,建議你可以簡單看過放在此資料夾裡的\href{run:./latex123.pdf}{李果正-大家來學\LaTeX}前四章內容,在下載適合的\LaTeX\ 整合發行套件之後(請看第~\ref{it:download}項),可以嘗試用剛安裝好的\LaTeX\ 編輯器來編譯\href{run:./thesis.tex}{thesis.tex}這份文件,編譯的方法可以看下面第~\ref{it:comp}項的介紹,若編譯成功,所編譯出來的thesis.pdf文件的應該會跟此demo.pdf文件一模一樣,而且沒有任何問號符號,走到這一步的話,就差不多可以開始邊學習\LaTeX\ 邊編輯你的畢業論文了!基本上會使用到的指令都包含在論文的的各章節裡,怎麼在論文裡寫公式或是放圖之類的就自行看tex檔學吧。如果有任何問題或建議可以來信與我討論,我的信箱是\href{mailto:dran31545@gmail.com}{dran31545@gmail.com},或是到此範本\href{http://code.google.com/p/ntu-thesis-latex-template/}{Google Project}裡面的\href{http://code.google.com/p/ntu-thesis-latex-template/issues/list}{Issues}貼上你的問題與建議,我會盡我所能更新此範本,也歡迎大家自行重製、改良此範本並散布給他人,祝大家順利畢業!\\\\
 要編輯致謝請打開\href{run:./acknowledgementsCH.tex}{acknowledgementsCH.tex}\\
 \begin{enumerate}[leftmargin=0pt, topsep=0pt, itemsep=0pt, label=\Roman{*}.]
\item 此範本參考並修改自下列網站的資料:
\begin{enumerate}[topsep=0pt, itemsep=0pt, label=$\bullet$]
    \item \href{http://www.csie.ntu.edu.tw/~tzhuan/www/resources/ntu/}{如何用 LaTeX 排版臺灣大學碩士論文}\\
    \textemdash 台灣大學論文\LaTeX\ 樣版原創者\href{http://www.csie.ntu.edu.tw/~tzhuan/www/}{黃子桓}的教學網頁
    \item \href{http://www.hitripod.com/blog/2012/05/latex-thesis-template-quick-reference/}{LaTeX 常用語法及論文範本}\\
    \textemdash \href{http://www.hitripod.com/blog/}{Hitripod}所修改的範本,這裡參考了許多他所寫的格式和內容
    \item \href{http://www.cc.ntu.edu.tw/chinese/epaper/0014/20100920_1404.htm}{使用LaTeX做出精美的論文}
    \item \href{http://www.hitripod.com/blog/2011/04/xetex-chinese-font-cjk-latex/}{XeTeX:解決 LaTeX 惱人的中文字型問題}
    \item \href{http://code.google.com/p/ntuthesis/}{台灣大學碩士、博士論文的Latex模板}\\
\end{enumerate}
\item 幾個有用的參考資料及網路資源:
\begin{enumerate}[topsep=0pt, itemsep=0pt, label=$\bullet$]
    \item \href{run:./latex123.pdf}{李果正-大家來學\LaTeX}\textemdash 建議先看完前四章
    \item \href{http://en.wikibooks.org/wiki/LaTeX}{WIKIBOOKS-\LaTeX}\textemdash 好用的線上工具書
    \item \href{run:./Working_with_a_bib_file_using_Jabref.pdf}{Working with a .bib file using JabRef}
    \item \href{run:./Fi087_S.pdf}{Using BibDesk - A short tutorial}
    \item \href{http://www.dfcd.net/articles/latex/latex.html}{LaTeX for Physicists}\\
\end{enumerate}
\item 下載\LaTeX\ 整合發行套件,可參考\href{http://www.tug.org/texcollection/}{TeX Collection}:\label{it:download}
 \begin{enumerate}[topsep=0pt, itemsep=0pt, label=\arabic{*}.]
     \item \href{http://www.tug.org/mactex/}{MacTeX}: For \textcolor{Green}{\textbf{MacOSX}},下載\href{http://mirror.ctan.org/systems/mac/mactex/MacTeX.pkg}{MacTeX.pkg}
     \item \href{http://www.tug.org/protext/}{ProTeXt}: For \textcolor{Green}{\textbf{Windows}},下載\href{ftp://ftp.fernuni-hagen.de/pub/windows/win32/ProTeXt/}{ISO file}
     \item \href{http://www.tug.org/texlive/}{TeX Live}: For \textcolor{Green}{\textbf{GNU/Linux}} and \textcolor{Green}{\textbf{MacOSX}}, and \textcolor{Green}{\textbf{Windows}},下載\href{http://www.tug.org/texlive/acquire-iso.html}{ISO file}
     \item \href{http://ctan.org/}{CTAN}: The Comprehensive TeX Archive Network.\\
 \end{enumerate}

\item 好用的程式:
 \begin{enumerate}[topsep=0pt, itemsep=0pt, label=$\bullet$]
    \item 文獻管理系統:
        \begin{enumerate}[topsep=0pt, itemsep=0pt, label=\arabic{*}.]
         \item \href{http://jabref.sourceforge.net/}{JabRef}\\
                     可參考\href{run:./Working_with_a_bib_file_using_Jabref.pdf}{Working with a .bib file using JabRef}或是\href{https://www.google.com/search?q=jabref}{Google}及\href{http://www.youtube.com/results?search_query=jabref}{YouTube}
         \item \href{http://bibdesk.sourceforge.net/}{BibDesk} (For Mac)\\
                     可參考\href{run:./Fi087_S.pdf}{Using BibDesk - A short tutorial}或是\href{https://www.google.com/search?q=bibdesk}{Google}及\href{http://www.youtube.com/results?search_query=bibdesk}{YouTube}
         \end{enumerate}
         \item 方程式編輯器:\href{https://chrome.google.com/webstore/detail/dinfmiceliiomokeofbocegmacmagjhe?hl=zh-TW}{Daum Equation Editor} (Chrome App,必須使用Google瀏覽器)\\
 \end{enumerate} 
\item 編譯流程:\label{it:comp}
\begin{enumerate}[topsep=0pt, itemsep=0pt, label=\arabic{*}.]
    \item \texttt{xelatex thesis}\\ 對thesis.tex進行第一次XeLaTeX編譯,產生thesis.pdf以其他檔案
    \item \texttt{bibtex thesis}\\ 對thesis.tex進行BibTeX編譯,產生bbl檔以及blg檔
    \item \texttt{xelatex thesis}\\ 對thesis.tex進行第二次XeLaTeX編譯,產生目錄、圖表連結及參考文獻
    \item \texttt{xelatex thesis}\\ 對thesis.tex進行第三次XeLaTeX編譯,產生參考文獻連結,完成編譯
\end{enumerate} 
    \textcolor{Red}{注意!}此範本使用cite套件,可依據你利用文獻管理系統所整理好的\href{run:./thesisbib.bib}{thesisbib.bib}檔在論文最後產生參考文獻頁面,若你的系所規定要在每個章節的後面產生參考文獻,則可以用chapterbib套件,來對每個有附參考文獻的章節tex檔進行一次BibTeX編譯產生bbl檔,如範例的\href{run:./introduction.tex}{introduction.tex}、\href{run:./THM.tex}{THM.tex}和\href{run:./EXP.tex}{EXP.tex},如果有這需要請把\href{run:./thesis.tex}{thesis.tex}檔裡使用cite套件的指令利用註解符號\texttt{\%}來取消使用cite套件,並刪去出現在使用chapterbib套件指令前面的註解符號\texttt{\%}來啟動使用chapterbib套件
    \begin{verbatim}
\usepackage{cite}
%\usepackage{chapterbib}
改成
%\usepackage{cite}
\usepackage{chapterbib}
    \end{verbatim}
    再來利用註解符號\texttt{\%}取消會把參考文獻放在論文最後的指令
    \begin{verbatim}
\bibliographystyle{unsrt}
\addcontentsline{toc}{chapter}{\bibname}
\bibliography{thesisbib}
改成
%\bibliographystyle{unsrt}
%\addcontentsline{toc}{chapter}{\bibname}
%\bibliography{thesisbib}
    \end{verbatim}
    再把用來輸入章節檔案的\texttt{\textbackslash input}指令改成\texttt{\textbackslash include}指令
     \begin{verbatim}
\input{introduction}  =>  \include{introduction}
\input{THM}           =>  \include{THM}
\input{EXP}           =>  \include{EXP}
     \end{verbatim}
    最後記得在每個有附參考文獻的章節加上產生參考文獻的指令,即在\href{run:./introduction.tex}{introduction.tex}、\href{run:./THM.tex}{THM.tex}和\href{run:./EXP.tex}{EXP.tex}三個檔案裡最後啟動下面兩行指令
     \begin{verbatim}
%\bibliographystyle{unsrt} => \bibliographystyle{unsrt}
%\bibliography{thesisbib}  =>  \bibliography{thesisbib}
     \end{verbatim}
    而編譯時則需要對有附參考文獻的\href{run:./introduction.tex}{introduction.tex}、\href{run:./THM.tex}{THM.tex}和\href{run:./EXP.tex}{EXP.tex}各做一次BibTeX 編譯,編譯流程如下
    \begin{enumerate}[topsep=0pt, itemsep=0pt, label=\arabic{*}.]
    \item \texttt{xelatex thesis}\\ 對thesis.tex進行第一次XeLaTeX編譯,產生thesis.pdf及其他檔案
    \item \texttt{bibtex introduction}\\ 對introduction.tex進行BibTeX編譯,產生bbl檔以及blg檔
    \item \texttt{bibtex THM}\\ 對THM.tex進行BibTeX編譯,產生bbl檔以及blg檔
    \item \texttt{bibtex EXP}\\ 對EXP.tex進行BibTeX編譯,產生bbl檔以及blg檔
    \item \texttt{xelatex thesis}\\ 對thesis.tex進行第二次XeLaTeX編譯,產生目錄、圖表連結及參考文獻
    \item \texttt{xelatex thesis}\\ 對thesis.tex進行第三次XeLaTeX編譯,產生參考文獻連結,完成編譯\\
\end{enumerate} 
\item 補充說明與注意事項:
\begin{enumerate}[topsep=0pt, itemsep=0pt, label=$\bullet$]
    \item 口試委員會審定書:\\
    請到台大圖書館網頁的\href{http://etds.lib.ntu.edu.tw/etdsystem/submit/submitLogin}{電子論文服務}下載\href{http://gra103.aca.ntu.edu.tw/gra2007/gra/tienn/\%E5\%AD\%B8\%E4\%BD\%8D\%E8\%80\%83\%E8\%A9\%A6\%E8\%A1\%A8\%E5\%86\%8A/THESISSAMPLE.DOC}{論文格式範本},並修改成正確的格式,也可到此範本所在資料夾的\href{run:./cert.doc}{cert.doc}修改。當然你也可以利用LaTeX來編輯,你只要填好\href{run:./ntuvars.tex}{ntuvars.tex}檔的資料,並去除在thesis.tex裡下面這行的註解符號\texttt{\%} 
    \begin{verbatim}
%\makecertification
    \end{verbatim}
    編譯完後就可以產生審定書格式。口試通過後,請把已經簽名的審定書掃描成pdf檔,再取代原本的\href{run:./cert.pdf}{cert.pdf},即可放上已簽名的審定書。處理審定書出現的指令在thesis.tex裡 
    \begin{verbatim}
%----------- generate the certification ...
%\makecertification
%----------- includepdf by using package ...
\addcontentsline{toc}{chapter}{口試委員會審定書}
\includepdf[pages={1}]{cert.pdf}
    \end{verbatim}
    \item 浮水印:\\
    資料夾已經附上浮水印檔案了,若學校有更改,到請到台大圖書館網頁的\href{http://etds.lib.ntu.edu.tw/etdsystem/submit/submitLogin}{電子論文服務}下載\href{http://etds.lib.ntu.edu.tw/files/watermark.pdf}{pdf格式的浮水印}到此範本所在資料夾。若要開啟關閉浮水印功能,即自行刪去或加上下面位於\href{run:./thesis.tex}{thesis.tex}指令的註解符號\texttt{\%}
    \begin{verbatim}
%\CenterWallPaper{0.174}{watermark.pdf}
%\setlength{\wpXoffset}{6.1725cm}
%\setlength{\wpYoffset}{10.5225cm}
    \end{verbatim}
    \item 單面印刷與雙面印刷:\\
    此範本為單面印刷,若論文頁數超過80頁,依規定需要用雙面印刷,此時只需把thesis.tex裡的
    \begin{verbatim}
\documentclass[a4paper, 12pt, oneside]{book}
改成
\documentclass[a4paper, 12pt, twoside]{book}
    \end{verbatim}
        \item 如何加入附錄?\\
    在\href{run:./thesis.tex}{thesis.tex}裡,依需求選擇input或include,刪去\texttt{\%}符號來輸入附錄章節
    \begin{verbatim}
%----------- Input your appendix here  -----------
%\chapter{First appendix title}

Open and edit \href{run:./AppendixA.tex}{AppendixA.tex}
%or %chapter cite  == \include
%\chapter{First appendix title}

Open and edit \href{run:./AppendixA.tex}{AppendixA.tex}
    \end{verbatim}
    在章節檔\texttt{AppendixA.tex}裡,開頭打
    \begin{verbatim}
\chapter{First appendix title}
    \end{verbatim}
    即可,以此類推。    
        \item 系上規定論文圖表須全部放到最後獨立出來的章節,且章節不出現在目錄中:\\
    在\href{run:./thesis.tex}{thesis.tex}裡,依需求選擇input或include,刪去\texttt{\%}符號來輸入圖表章節
    \begin{verbatim}    
%----------- Input your Figure chapter here  -----------
%\input{EndFigTab} 
%chapter cite  == \include
%\include{EndFigTab}
    \end{verbatim}
    在章節檔\href{run:./EndFigTab.tex}{EndFigTab.tex}裡有範例和說明可供參考,要注意正文的圖表和附錄的圖表要分清楚,即在\href{run:./EndFigTab.tex}{EndFigTab.tex}內
    \begin{verbatim}    
\renewcommand{\thefigure}{\arabic{chapter}.\arabic{figure}} 
\renewcommand{\thetable}{\arabic{chapter}.\arabic{table}} 
%--- Input your main figures and tables here  ---
    \end{verbatim}
    這幾行之後章節計數器格式已切換為1\dots 9,放正文的圖表 ,
     \begin{verbatim}    
\renewcommand{\thefigure}{\Alph{chapter}.\arabic{figure}} 
\renewcommand{\thetable}{\Alph{chapter}.\arabic{table}}
%--- Input your appendix figures and tables here  ---
    \end{verbatim}
    這幾行之後章節計數器格式已切換為A\dots Z,放附錄的圖表。另外要取消圖表的浮動功能,才能讓圖表按照指令出現順序排好,即把平常使用的圖表指令
    \begin{verbatim}    
\begin{figure}[htb]
...
\begin{table}[htb]
    \end{verbatim}
    改成
     \begin{verbatim}    
\begin{figure}[!]
...
\begin{table}[!]
    \end{verbatim}
    剩下的只要注意章節圖表的計數器設定即可。\texttt{\textbackslash ref}和\texttt{\textbackslash label}指令可以在此圖表章節與正文章節使用。
     \item 如果我想要修改margin(文字邊界)的話,可以從哪裡下手呢?\\
     請打開\href{run:./ntu.sty}{ntu.sty}修改下面這行的上下左右參數即可:
    \begin{verbatim}
\RequirePackage[top=3cm,left=3cm,bottom=2cm,right=3cm]{geometry}
    \end{verbatim}
    \item 我想引用Twomey (1974): Pollution and planetary albedo這篇論文,如何用\texttt{\textbackslash cite}引用它的時候在內文顯示Twomey (1974) [編號] ?\\
    建議使用natbib套件,參考資料如下:\\
    \href{http://en.wikibooks.org/wiki/LaTeX/Bibliography_Management}{LaTeX/Bibliography Management}\\
    \href{http://nodonn.tipido.net/bibstyle.php}{Overview of Bibtex-Styles}\\
    \href{http://merkel.zoneo.net/Latex/natbib.php}{Reference sheet for natbib usage }\
 \item \XeTeX\ :\\
    此範本中文字體使用\XeTeX\ 轉換,細節請參考\href{http://www.hitripod.com/blog/}{Hitripod}寫的\href{http://www.hitripod.com/blog/2011/04/xetex-chinese-font-cjk-latex/}{ 
XeTeX:解決 LaTeX 惱人的中文字型問題}。
 \item 如何輸入英文`單引號'和``雙引號''以及不同長度的破折號?\\
        可以參考\href{run:./latex123.pdf}{李果正-大家來學\LaTeX}第17頁針對標點符號的遊戲規則,範例如下,輸入以下指令:\\
        \begin{verbatim}
`單引號'\\
``雙引號''\\
-hyphen\\
--en-dash\\
---em-dash\\
        \end{verbatim} 
        則顯示:\\
       `單引號'\\
        ``雙引號''\\
        -hyphen\\
        --en-dash\\
        ---em-dash\\
    \end{enumerate} 
    \end{enumerate} 
               

%----------- Have a fractal fern? -----------
%\begin{pspicture}
%\psFern[scale=30,maxIter=100000,linecolor=Green]
%\end{pspicture}

 \end{acknowledgementsCH}