\documentclass[12pt,a4paper]{article}

% packages
\usepackage{CJKutf8}  % 使用CJK utf8字元(令有gbk)
\usepackage{amsmath, amsfonts, amssymb}
\usepackage{color, enumerate}
\usepackage{bm}
\usepackage{graphicx}
\usepackage{picinpar,epic}
\usepackage{subfigure}
\usepackage[margin=2cm]{geometry}
\usepackage{multirow} 
\usepackage{longtable}
\usepackage{cases}

% line spacing 
\newlength{\defbaselineskip}
\setlength{\defbaselineskip}{\baselineskip}
\newcommand{\setlinespacing}[1]{\setlength{\baselineskip}{#1 \defbaselineskip}}
\newcommand{\halfspacing}{\setlength{\baselineskip}{0.30 \defbaselineskip}}
\newcommand{\singlespacing}{\setlength{\baselineskip}{1.20 \defbaselineskip}}
\newcommand{\oneandahalfspacing}{\setlength{\baselineskip}{1.33 \defbaselineskip}}
\newcommand{\doublespacing}{\setlength{\baselineskip}{1.67 \defbaselineskip}}

\renewcommand{\arraystretch}{1.5}

% commands
\newcommand{\bitem} {\begin{itemize}}
\newcommand{\eitem} {\end{itemize}}
%\renewcommand{\vec}{\mathbf} % vector is bold, non-italic 




%% \documentclass[12pt,a4paper]{article}
%% \usepackage{CJKutf8}
%
%% line spacing
%\newlength{\defbaselineskip}
%\setlength{\defbaselineskip}{\baselineskip}
%\newcommand{\setlinespacing}[1]{\setlength{\baselineskip}{#1 \defbaselineskip}}
%\newcommand{\halfspacing}{\setlength{\baselineskip}{0.30 \defbaselineskip}}
%\newcommand{\singlespacing}{\setlength{\baselineskip}{1.20 \defbaselineskip}}
%\newcommand{\oneandahalfspacing}{\setlength{\baselineskip}{1.33 \defbaselineskip}}
%\newcommand{\doublespacing}{\setlength{\baselineskip}{1.67 \defbaselineskip}}


\begin{document}
\begin{CJK}{UTF8}{bkai} % 開始 CJK 環境,編碼,字體
\title{ Group Meeting Comments for 蘇庭玉}
\date{September 23, 2013}
\author{Kuei-Yuan Chan\footnote{optional, you can do anonymous comments}}

\maketitle

\section{General comments for this research}
\begin{enumerate}[(a)]
\item 目前的報告內容仍十分缺乏研究動機及目的,且所有主題圍繞著停車系統,但妳的論文要展現的是更多研究的深度,請不要把論文變成專利說明書。我的建議是重新製作新的投影片,不要被現存的投影片所侷限,仔細想想以下問題:
\begin{quote}
How to reduce/prevent human uncertainty in a goal to achieve higher product reliability ? 
\end{quote}
\item 請問你一整個暑假的文獻探討內容為何?這一部分是堆疊出妳的研究主旨的重要方針,沒有完整的文獻,所有的研究只變成單純的`我想要這樣做',或是`別人要我這樣做'。請不要把現(既)有的東西硬塞進一個想像的題目而完成一份報告,想想研究單位工讀生與研究員的差別。
\item Think about how to transit from parking problem to general human-machine reliability problem.
\end{enumerate}

\section{General comments about this presentation}
\begin{enumerate}[(a)]
\item Can the mouse move with the presentation in the final file?
\item Can we forward/backward at any place of the slide when we listen to your talk? Can we fast forward ?
\item 報告的動畫過多,導致很多資訊在播放該張後便消失。
\item 若自行變換投影片,則與口白不同步,如何處理?
\item Too much talking without visual aids.
\end{enumerate}

\section{Specific comments about this presentation}
\begin{enumerate}[(a)]
\item ARX model needs some example. The way you presented only finished your task, not intended for people to understand.
\end{enumerate}

\end{CJK} % 結束 CJK 環境
\end{document}
