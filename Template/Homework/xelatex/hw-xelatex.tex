\documentclass[12pt, a4paper]{article}

\usepackage{xeCJK}
\usepackage[margin=2.0cm]{geometry}
\usepackage{fancyhdr}
\usepackage{graphicx}
\usepackage{amsmath, amssymb, amsfonts}
\usepackage{listings} % code showing
\usepackage{pagecolor} % for code color define
\usepackage{tocloft}

% -- 字體 (注意檔案是否存在)
\setCJKmainfont[Path="./fonts/"]{kaiu}
\setmainfont[
    Path="./fonts/",
    BoldFont={Times New Roman Bold},
    ItalicFont={Times New Roman Italic},
    BoldItalicFont={Times New Roman Bold Italic}
]{Times New Roman}

% -- 段落格式
\setlength{\headheight}{28pt}
\pagestyle{fancy}
\fancyhf{}
\setlength\parindent{0pt}
\setlength{\parskip}{0.5em}
\setcounter{secnumdepth}{-1}
\renewcommand{\cftsecleader}{\cftdotfill{\cftdotsep}}

% -- 圖, 表標題使用中文取代
\renewcommand{\figurename}{圖}
\renewcommand{\tablename}{表}

% -- 圖、表標題與圖表之間距離
\setlength{\abovecaptionskip}{10pt}
\setlength{\belowcaptionskip}{10pt}

% \renewcommand{\section}[2]{}

% -- 顯示程式碼格式
\definecolor{codegreen}{rgb}{0,0.6,0}
\definecolor{codegray}{rgb}{0.5,0.5,0.5}
\definecolor{codepurple}{rgb}{0.58,0,0.82}
\definecolor{backcolour}{rgb}{0.95,0.95,0.92}
\lstdefinestyle{pystyle}{
    backgroundcolor=\color{backcolour},
    commentstyle=\color{codegreen},
    keywordstyle=\color{magenta},
    numberstyle=\footnotesize\color{codegray},
    stringstyle=\color{codepurple},
    basicstyle=\ttfamily\footnotesize,
    breakatwhitespace=false,
    breaklines=true,
    captionpos=b,
    keepspaces=true,
    numbers=left,
    numbersep=5pt,
    showspaces=false,
    showstringspaces=false,
    showtabs=false,
    tabsize=2,
    extendedchars=false
}
\lstset{style=pystyle}

% ---------------------------------------------

\begin{document}

\chead{\textbf{Homework 1}}
\lhead{課程名稱}
\rhead{姓名 1\\ 姓名 2}
\cfoot{\thepage}

% \ ~ \

\tableofcontents\thispagestyle{fancy}

\section{Problem 1}

測試測試測試測試測試測試測試測試測試測試測試測試測試測試測試測試測試測試測試測試測試測試測試測試測試測試測試測試測試測試測試測試測試測試測試測試測試測試測試測試測試測試測試測試測試測試測試測試測試測試測試測試測試測試測試測試測試

\subsection*{Solution}
測試測試測試測試測試測試測試測試測試測試測試測試測試測試測試測試測試測試測試測試測試測試測試測試測試測試測試測試測試測試測試測試測試測試測試測試測試測試測試測試測試測試測試測試測試測試測試測試測試測試測試測試測試測試測試測試測試測試
\begin{equation*}
    a = \sum_{i=1}^{10} k_i
\end{equation*}
測試測試測試測試測試測試測試測試測試測試測試測試測試測試測試測試測試測試測試測試測試測試測試測試測試測試測試測試測試測試測試測試測試測試測試測試測試測試測試測試測試測試測試測試測試測試測試測試測試測試測試測試測試測試測試測試測試測試

測試測試測試測試測試測試測試測試測試測試測試測試測試測試測試測試測試測試測試測試測試測試測試測試測試測試測試測試測試測試測試測試測試測試測試測試測試測試測試測試測試測試測試測試測試測試測試測試測試測試測試測試測試測試測試測試測試測試

\clearpage
\section{Problem 2}

\lstinputlisting[language=Python, firstline=5, lastline=7, firstnumber=5]{./code.py}
\lstinputlisting[language=Python, caption=demo, label=code:dwmo]{./code.py}

根據文獻 \cite{Chen2017b}

根據文獻 \cite{Features2018b}

根據文獻 \cite{Szeliski2010b}

\clearpage
\section{Problem 3}

\clearpage
\section{Problem 4}

\clearpage

\section{References}
% -- 隱藏預設參考文獻標題
\begingroup
    \renewcommand{\section}[2]{}
    \bibliographystyle{ieeetr}
    \bibliography{ref.bib}
\endgroup

\end{document}
