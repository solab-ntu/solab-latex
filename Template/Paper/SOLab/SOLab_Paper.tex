\documentclass[a4paper,10pt]{article}

% packages
\usepackage{amsmath, amsfonts, bm, cite}
\usepackage[pdftex]{graphicx}
\usepackage{color, subfigure}
\usepackage{multirow}
\usepackage{multicol}
\usepackage{makeidx, chngpage}

% line spacing
\newlength{\defbaselineskip}
\setlength{\defbaselineskip}{\baselineskip}
\newcommand{\setlinespacing}[1]{\setlength{\baselineskip}{#1 \defbaselineskip}}
\newcommand{\halfspacing}{\setlength{\baselineskip}{0.30 \defbaselineskip}}
\newcommand{\singlespacing}{\setlength{\baselineskip}{1.20 \defbaselineskip}}
\newcommand{\oneandahalfspacing}{\setlength{\baselineskip}{1.33 \defbaselineskip}}
\newcommand{\doublespacing}{\setlength{\baselineskip}{1.67 \defbaselineskip}}

\renewcommand{\arraystretch}{1.5}

% commandsa
\newcommand{\muX}{\bm{\mu}_{\mathbf{X}}}
\newcommand{\bitem}{\begin{itemize}} 
\newcommand{\eitem}{\end{itemize}}
\newcommand{\be}{\begin{equation}}
\newcommand{\ee}{\end{equation}}
\newcommand{\ba}{\begin{eqnarray}}
\newcommand{\ea}{\end{eqnarray}}
\newcommand{\nn}{\notag}
\renewcommand{\vec}{\mathbf} % vector is bold, non-italic 

%\renewcommand{\baselinestretch}{2} % double spacing

\begin{document}

\begin{center}~\\~\\

{\Large {\sc SOLab Paper Template}}\\~\\

\singlespacing
Michael Jordan\footnote[1]{Graduate Student}, Kuei-Yuan Chan\footnote[2]{Corresponding Author, Associate Professor, Fax:+886-2-2363-1755}, \\
 {\tt chanky@ntu.edu.tw}\\
Department of Mechanical Engineering, \\
National Taiwan University, Taipei, Taiwan\\~\\

%Submitted for Publication in the {\em Journal of Mechanical Design}%\\ Special Issue on Design Under Uncertainty

\doublespacing
\end{center}


%%%%%%%%%%%%%%%%%%%%%%%%%%%%%%%%%%%%%%%%%%%%%%%%%%%
\begin{abstract}
Here is where the abstract should be. In general, abstract has only one paragraph with no equations and figures. 
\end{abstract}
~\\~\\
\textbf{Keywords} : put some keywords that you think are relevant to your work, keyword 1, keyword 2\newpage
\noindent \textbf{Nomenclature}
\begin{description}
\item[$a_0$] the average rate of change of the objective function.
\item[$a_j$] the average rate of change of the $j$th constraint function.
\item[$c_j$] the weight coefficient of the $j$th inequality constraint.
\item[$d$] the size of the subspace, the distance between center and vertex, a half of diagonal.
\item[$d_{\mathrm{c},i}^{k}$] the size of the $k$th subspace from $i$th  parent space.
\item[$f_*$ or $f_{\mathrm{min}}$] current best function value.
\item[$f_{\mathrm{c},i}^{k}$] sampling the $k$th result by SQP from $i$th parent space in S.A. DIRECT.
\item[$g_j^r$] the violation value of the $r$th sub space violate the $j$th constraint.
\item[$i$] the dummy number of design variables($i=1,\cdots n$).
\item[$j$] the dummy number of the constraint($j=1,\cdots m$).
\item[$K$] tuning parameter.
\item[$l_i$] the $i$th design variables lower bound.
\item[$lb_i$] the $i$th lower boundary.
\item[$m$] the number of all constraint.
\item[$n$] the dimension(or number) of the design variables.
\item[$\mathcal{S}_{\mathrm{p}, i}$] the $i$th parent space, selected from all subspace in S.A. DIRECT.
\item[$\mathcal{S}_{\mathrm{c}, i}^k$] the $k$th subspace(or child space), produced from the $i$th parent space in S.A. DIRECT.
\item[$s_0$] the sum of observed rates of change of the objective function.
\item[$s_j$] the sum of observed rates of change of the $j$th constraint function.
\item[$u_i$]   the $i$th upper bound.
\item[$ub_i$]   the distance between sample and the $i$th upper bound.
\item[$x_r$] sampling point by DIRECT algorithm. 
\item[$\vec{x}_{\mathrm{c},i}^{k}$] the $k$th sampling point by SQP from $i$th parent space in S.A. DIRECT.
\item[$\theta$] the relation of global and local with respect to current optimum.
\item[$\varepsilon$] balance parameter, adjusting the process of selecting, avoiding to selecting the subspace too small.
\item[$\epsilon$] $\varepsilon \times f_{min}$,  the concept likes $\varepsilon$.
\end{description}
\newpage

%%%%%%%%%%%%%%%%%%%%%%%%%%%%%%%%%%%%%%%%%%%%%%%%%%%
\section{Introduction}

\section{Literature Review}

\section{Methodology}


\section{Engineering case study : design of a belt-pulley mechanism}

\bibliography{}
\bibliographystyle{unsrt}
\end{document}



