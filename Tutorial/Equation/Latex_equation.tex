\documentclass[12pt,a4paper]{report}

% packages
\usepackage{CJKutf8}
\usepackage{amsmath, amsfonts, amssymb}
\usepackage{indentfirst}
\usepackage[margin=2.5cm]{geometry}
\usepackage{titlesec}
\usepackage{framed}
%\usepackage[]{mcode}

% % line spacing 

\newlength{\defbaselineskip}
\setlength{\defbaselineskip}{\baselineskip}
\newcommand{\setlinespacing}[1]{\setlength{\baselineskip}{#1 \defbaselineskip}}
\newcommand{\halfspacing}{\setlength{\baselineskip}{0.30 \defbaselineskip}}
\newcommand{\singlespacing}{\setlength{\baselineskip}{1.20 \defbaselineskip}}
\newcommand{\oneandahalfspacing}{\setlength{\baselineskip}{1.33 \defbaselineskip}}
\newcommand{\doublespacing}{\setlength{\baselineskip}{1.67 \defbaselineskip}}


\renewcommand{\arraystretch}{1.5}

% commands
\renewcommand{\arraystretch}{1.5}
\renewcommand{\figurename}{圖}
\renewcommand{\tablename}{表}
\renewcommand\contentsname{目錄}
\renewcommand\listfigurename{圖目錄}
\renewcommand\listtablename{表目錄}


\begin{document}
\begin{CJK}{UTF8}{bkai}
\titleformat{\chapter}{\Huge}{\textbf{第 \thechapter\ 章}}{1em}{\textbf}
\thispagestyle{empty}
\begin{center}
       ~\\
        \vspace{6.8cm}

        \textbf{\Huge
國立台灣大學機械工程學系 \\
系統最佳化實驗室
}
        \vspace{3cm}

        \textbf{\Huge
	Latex Equation and Debug
        }
        \vspace{6cm}

        {\large
        By  Tzu-Chieh Hung
        }
        \vspace{4cm}

        {\large
            03/20/2014
        }
    \end{center}

\newpage

%\tableofcontents
%\listoffigures


\thispagestyle{empty}
~
\newpage
\clearpage
\setcounter{page}{1}
\chapter{數學排版}

在 \LaTeX 的數學模式中,所有的文字、符號預設都為斜體字,如果要在數學模式中插入一般的正常文字,需要使用 \verb|\mbox{}| 或 \verb|\textmr{}| 將文字包圍起來。除了字型會被自動調整之外,所有的空白也都會被忽略。如果需要插入空白,可以使用調整橫向空間的相關指令,如: \verb|\hspace{}| 、 \verb|\quad| 、 \verb|\thinspace| 等指令進行調整。

\LaTeX 中的數學模式可分為隨文模式(math in line mode)與展式模式(math display mode)兩種。隨文模式用在文字與數學式混合在一起的排版;展式模式則會將數學式單獨形成一行,且會與正常文字有一定的空間區隔。

\section{隨文模式(math in line mode)}
\noindent以下三種方法可以直接進入隨文數學模式:
\begin{enumerate}
	
	\item \verb|$ 數學式 $|:這個方法通常用來插入簡單的數學式 $f(x)$ 或符號 $\mu + \sigma$ 。
		\begin{framed}
			\begin{verbatim}
			這個方法通常用來插入簡單的數學式 $f(x)$ 或符號 $\mu + \sigma$ 。
			\end{verbatim}
		\end{framed}
	
	\item \verb|\begin{math} 數學式 \end{math}|:如果遇到比較長的數學式,則可以改用環境的方式來輸入,不過要注意的是在這個環境的上下不能留空白行,否則它會另起段落,這樣就失去隨文模式的意義了。
		\begin{math}
		f(\mathrm x) = x_1 + x_2 + x_3 + x_4 + x_5 + x_6 + x_7 + x_8 + x_9 + \cdots
		\end{math}。 各位可以看到在隨文數學式中,\LaTeX 會自動換行。
		\begin{framed}
			\begin{verbatim}
			如果遇到比較長的數學式,則可以改用環境的方式來輸入,不過要注意的是在這個環境的上下不能留空白行,否則它會另起段落,這樣就失去隨文模式的意義了。
				\begin{math}
				f(\mathrm x) = x_1 + x_2 + x_3 + x_4 + x_5 + x_6 + x_7 + x_8 + x_9
				 + \cdots
				\end{math}。 各位可以發現在隨文數學式中,\LaTeX 會自動換行。
			\end{verbatim}
		\end{framed}
	
	\item \verb|\( 數學式 \)|:這是\verb|\begin{math} 數學式 \end{math}| 的省略寫法。在隨文數學式中,數學式的整體高度會被限制為字體高度,如果要取消限制,可以使用 \verb|\displaystyle| 強制顯示為展式模式。\( f(x,\mu,\sigma) = \frac{1}{\sigma \sqrt{2\pi}}\displaystyle\exp\left[-\frac{1}{2}(\frac{x-\mu}{\sigma})^2\right]\)。
		\begin{framed}
			\begin{verbatim}
			\( 
			f(x,\mu,\sigma) = \frac{1}{\sigma \sqrt{2\pi}}
			\displaystyle\exp\left[-\frac{1}{2}(\frac{x-\mu}{\sigma})^2\right]
			\)
			\end{verbatim}
		\end{framed}
\end{enumerate}

\section{展式模式(math display mode)}
展式數學式通常用來描述獨立與比較複雜的數學式,它會讓數學式獨立成為一行,且會使用較大的數學符號與字體,有需要的話,也可以加入編號。

以下三種方法可以進入一般的展式模式:
\begin{enumerate}
\item \verb|\begin{displaymath} 數學式 \end{displaymath}|
	\begin{displaymath}
	f(x,\mu,\sigma) = \frac{1}{\sigma \sqrt{2\pi}}\exp\left[-\frac{1}{2}(\frac{x-\mu}{\sigma})^2\right] 
	\end{displaymath}
	\begin{framed}
		\begin{verbatim}
			\begin{displaymath}
			f(x,\mu,\sigma) = \frac{1}{\sigma \sqrt{2\pi}}
			\exp\left[-\frac{1}{2}(\frac{x-\mu}{\sigma})^2\right] 
			\end{displaymath}
		\end{verbatim}
	\end{framed}
\item \verb|\[ 數學式 \]| 這是 \verb|\begin{displaymath} 數學式 \end{displaymath}| 的省略寫法。這兩種展式數學式都不會有方程式編號。

\item \verb|\begin{equation} 數學式 \end{equation}|:這是有方程式編號的數學式。如果要取消方程式編號,則可以使用 \verb|equation*| 。
	\begin{equation}
		f(x,\mu,\sigma) = \frac{1}{\sigma \sqrt{2\pi}}\exp\left[-\frac{1}{2}(\frac{x-\mu}{\sigma})^2\right] 
	\end{equation}
	\begin{framed}
		\begin{verbatim}
			\begin{equation}
			f(x,\mu,\sigma) = \frac{1}{\sigma \sqrt{2\pi}}
			\exp\left[-\frac{1}{2}(\frac{x-\mu}{\sigma})^2\right] 
			\end{equation}
		\end{verbatim}
	\end{framed}
\end{enumerate}

\section{矩陣與多行方程式}
除了最簡單的展式模式之外,我們還常會用到矩陣與多行的方程式,在此將逐一介紹。

\subsection{矩陣}
沒有外括號的矩陣可以透過 \verb|\begin{array} 矩陣內容 \end{array}| 來達成,矩陣內部的寫法與表格寫法類似,都是使用 \& 將各個欄位分開。
	\begin{equation}
		\begin{array}{ccc}
		a & b & c \\
		d & e& f \\
		g & h & i
		\end{array}
		\label{eq:array}
	\end{equation}
	
	\begin{framed}
	\begin{verbatim}
		\begin{equation}
		\begin{array}{ccc}
		a & b & c \\
		d & e& f \\
		g & h & i
		\end{array}
	\end{equation}
	\end{verbatim}
	\end{framed}
	在(\ref{eq:array})中,如果需要加上左右括號,可以使用 \verb|\left[ \right)| 將 \verb|array| 環境包圍起來,如(\ref{eq:leftright})。值得注意的是,由於大括號為 \LaTeX 內建指令,所以如果要使用大括號,則需要在前面加上一個 \textbackslash;而如果只需要單邊括號,另一邊則以英文句點取代。
	\begin{equation}
	\left\{
	\begin{array}{ccc}
	a & b & c \\
	d & e& f \\
	g & h & i
	\end{array}
	\right]
	\cdot
	\left(
	\begin{array}{ccc}
	a & b & c \\
	d & e& f \\
	g & h & i
	\end{array}
	\right. 
	\label{eq:leftright}
	\end{equation}
	\begin{framed}
	\begin{verbatim}
	\begin{equation}
		\left\{
		\begin{array}{ccc}
		a & b & c \\
		d & e& f \\
		g & h & i
		\end{array}
		\right]
		\cdot
		\left(
		\begin{array}{ccc}
		a & b & c \\
		d & e& f \\
		g & h & i
		\end{array}
		\right. 
	\end{equation}
	\end{verbatim}
	\end{framed}
使用上述方法,可以簡單的寫出分段方程式:
\[
f(x) = 
\left\{
	\begin{array}{ccc}
	ax+b & , & x\ge 0 \\
	cx+d & , & x< 0 
	\end{array}
\right.
\]
\begin{framed}
\begin{verbatim}
\[
f(x) = 
\left\{
	\begin{array}{ccc}
	ax+b & , & x\ge 0 \\
	cx+d & , & x< 0 
	\end{array}
\right.
\]
\end{verbatim}
\end{framed}
除了 \verb|array| 環境之外, \verb|matrix| 環境也提供了許多有分界符號的矩陣。
	\[
	\begin{matrix}
	a & b \\
	c & d
	\end{matrix}
	\begin{pmatrix}
	a & b \\
	c & d
	\end{pmatrix}
	\begin{bmatrix}
	a & b \\
	c & d
	\end{bmatrix}
	\begin{Bmatrix}
	a & b \\
	c & d
	\end{Bmatrix}
	\begin{vmatrix}
	a & b \\
	c & d
	\end{vmatrix}
	\begin{Vmatrix}
	a & b \\
	c & d
	\end{Vmatrix}
	\]
	\begin{framed}
	\begin{verbatim}
	\[
	\begin{matrix}
	a & b \\
	c & d
	\end{matrix}
	\begin{pmatrix}
	a & b \\
	c & d
	\end{pmatrix}
	\begin{bmatrix}
	a & b \\
	c & d
	\end{bmatrix}
	\begin{Bmatrix}
	a & b \\
	c & d
	\end{Bmatrix}
	\begin{vmatrix}
	a & b \\
	c & d
	\end{vmatrix}
	\begin{Vmatrix}
	a & b \\
	c & d
	\end{Vmatrix}
	\]
	\end{verbatim}
	\end{framed}
	
\subsection{多行方程式}
 \verb|align| 環境可以用來輸入多行方程式,在 \verb|align| 環境中,可以使用 \verb|&| 將特定位置對齊,以我們常用的最佳化方程式來說:
\begin{align}
\min_x \quad & f(x) \\
\mbox{s.t.} \quad & g(x) \le 0\\
& h(x) = 0 
\end{align}
\begin{framed}
\begin{verbatim}
\begin{align}
\min_x \quad & f(x) \\
\mbox{s.t.} \quad & g(x) \le 0\\
& h(x) = 0 
\end{align}
\end{verbatim}
\end{framed}

在上述方程組中, $f(x)$ 、 $g(x)$ 與 $h(x)$ 透過前面的 \verb|&| 符號而對齊。在此方程組中,我們通常會希望方程式標號是一組的,而不是分開為三個標號,這時候就可以使用 \verb|subequations| 環境將整個 \verb|align| 環境包起來,並透過不同的 \verb|label| 來呼叫方程式。如(\ref{eq:opt})表示的就是整個方程組,而(\ref{eq:obj})就是指目標函數。此外,若不想要某行方程式的標號,則只要在該行方程式後面加上 \verb|\notag| 指令即可。\begin{subequations}
\begin{align}
\min_x \quad & f(x) \label{eq:obj}\\
\mbox{s.t.} \quad & g(x) \le 0 \notag\\
& h(x) = 0 
\end{align}
\label{eq:opt}
\end{subequations}

\begin{framed}
\begin{verbatim}
(\ref{eq:opt})表示的就是整個方程組,而(\ref{eq:obj})就是指目標函數。
\begin{subequations}
\begin{align}
\min_x \quad & f(x) \label{eq:obj}\\
\mbox{s.t.} \quad & g(x) \le 0  \notag \\
& h(x) = 0 \notag
\end{align}
\label{eq:opt}
\end{subequations}
\end{verbatim}
\end{framed}
另外一種情況是我們並不需要方程式的子標號,這時候只要在 \verb|align| 環境中再使用 \verb|split| 環境包起來即可。
\begin{align}
\begin{split}
\min_x \quad & f(x) \\
\mbox{s.t.} \quad & g(x) \le 0\\
& h(x) = 0 
\end{split}
\end{align}

\begin{framed}
\begin{verbatim}
\\begin{align}
\begin{split}
\min_x \quad & f(x) \\
\mbox{s.t.} \quad & g(x) \le 0\\
& h(x) = 0 
\end{split}
\end{align}
\end{verbatim}
\end{framed}

\chapter{編譯與 Debug}

在一般的文件中, \LaTeX 僅需要編譯 .tex 檔所以並不會有太大的問題。但在論文寫作的過程中,由於加入了參考文獻的 .bib 檔,所以常會出現編譯錯誤或是參考文獻與 .bib 檔案內容不合的情況,因此在這裡簡述整個 \LaTeX 的編譯過程,若各位在編譯過程中遇到問題,請依循以下步驟進行編譯:
\begin{enumerate}
\item pdfLaTeX: 在第一次的編譯過程中,\LaTeX 會截取 .tex 檔內的各種 \verb|\cite{}| 與 \verb|\label{}|,因此在經過第一次編譯之後,所有文章中呼叫的方程式、圖片與表格都會以\ref{}呈現,而參考文獻則是以\cite{unknown}呈現。
\item BibTeX: 如果有使用到參考文獻的 .bib 檔,則需要進行這一個步驟,讓程式在 .bib 檔內抓取相關的資料。
\item pdfLaTeX: 在第二次編譯過程中, \LaTeX 已經有參考文獻的相關資料,但是還不確定其順序,所以仍是以\cite{unknown}呈現;而在方程式、圖、表的部分,因為在前一次的編譯中已經有圖表的出現順序,所以會完整呈現圖表的編號。
\item pdfLaTeX:最後一次編譯會將原本呈現\cite{unknown}的參考文獻顯示為正確的編號,在文章最後面也會依序條列出參考文獻。
\end{enumerate}

\section{常見問題與處理方式}

\noindent在 \LaTeX 中,較常見且通常看不出原因的問題有兩種:
\begin{enumerate}
\item 編譯錯誤一次後,怎麼修改甚至改回之前編譯正確的檔案, \LaTeX 仍然無法編譯:

針對這個問題,主要是發生在 .aux 檔案因為編譯失敗中止而沒有回復到正常的狀態,所以只要在 File 選單中選擇 Remove Aux Files 或是直接進資料夾刪除 \LaTeX 自動產生的檔案後,再重新編譯即可恢復正常。
\item 修改 .bib 檔案之後, \LaTeX 可以正常編譯,但是輸出的參考文獻與 .bib 檔案內的參考文獻資料不同:

這個問題通常是發生在新增或更新 .bib 檔內的資料之後,發現無論怎麼做,編譯出來的 .pdf 都還是原本的資料。這是因為 \LaTeX 在執行的過程中由 .bib 檔產生了一個 .bbl 檔,而在編譯的時候,參考文獻其實是從 .bbl 檔抓取的,所以只要去資料夾內刪除 .bbl 檔,然後再依照上述的編譯步驟編譯一次即可。 
\end{enumerate}

\begin{framed}
\noindent\textbf{萬用解:}刪除資料夾內,系統自動產生的所有檔案,然後依照上述過程重新編譯。如果還是無法編譯,那就是\textbf{你寫錯了!}
\end{framed}

\clearpage
\end{CJK}
\end{document}
