\documentclass[12pt,a4paper]{report}

% packages
\usepackage{CJKutf8}
\usepackage{amsmath, amsfonts, amssymb}
\usepackage{color}
\usepackage{bm}
\usepackage{graphicx}
\usepackage{picinpar,epic}
\usepackage{subfigure}
\usepackage{titlesec}
\usepackage{indentfirst}
\usepackage[margin=2cm]{geometry}
\usepackage{cases}
\usepackage{multirow} 

\usepackage{longtable}
\usepackage{listings}
\usepackage{slashbox}

\usepackage{array}
\usepackage{rotating,booktabs}


\makeatletter
\def\hlinewd#1{
\noalign{\ifnum0=`}\fi\hrule \@height #1
\futurelet\reserved@a\@xhline}
\makeatother

\newcolumntype{k}{!{\vrule width 3pt}}

% \newcommand{\vlinewd#1}{\vrule width #1}




% % line spacing 

\newlength{\defbaselineskip}
\setlength{\defbaselineskip}{\baselineskip}
\newcommand{\setlinespacing}[1]{\setlength{\baselineskip}{#1 \defbaselineskip}}
\newcommand{\halfspacing}{\setlength{\baselineskip}{0.30 \defbaselineskip}}
\newcommand{\singlespacing}{\setlength{\baselineskip}{1.20 \defbaselineskip}}
\newcommand{\oneandahalfspacing}{\setlength{\baselineskip}{1.33 \defbaselineskip}}
\newcommand{\doublespacing}{\setlength{\baselineskip}{1.67 \defbaselineskip}}


\renewcommand{\arraystretch}{1.5}

% commands
\renewcommand{\arraystretch}{1.5}
\renewcommand{\figurename}{圖}
\renewcommand{\tablename}{表}
\renewcommand\contentsname{目錄}
\renewcommand\listfigurename{圖目錄}
\renewcommand\listtablename{表目錄}


\begin{document}
\begin{CJK}{UTF8}{bkai}
\titleformat{\chapter}{\Huge}{\textbf{第 \thechapter\ 章}}{1em}{\textbf}
\thispagestyle{empty}
\begin{center}
       ~\\
        \vspace{6.8cm}

        \textbf{\Huge
國立台灣大學機械工程學系 \\
系統最佳化實驗室}

        \vspace{3cm}

        \textbf{\Huge
	Latex 表格製作簡述
        }
        \vspace{11.5cm}

        {\large
            06/03/2014
        }
    \end{center}

\newpage

\tableofcontents
%\listoffigures
\listoftables

\thispagestyle{empty}
~
\newpage

\chapter{簡易表格應用}

\section{簡易表格}
當文件中需要使用普通表格(不超過頁面)時,可使用以下指令製表,結果如下表 1.1,注意要點如下:
\begin{itemize}
    \item 程式碼第一的 [h] 為強制此表格的相對位置,若不輸入此命令,表格會被自動放置於頁面最上或最下方。
    \item 程式碼第五行 $| l | c | c |$ 代表第一欄靠左對齊,第二三欄置中對齊。
    \item 表號指令 "\textbackslash caption"不可寫在 "\textbackslash end\textbraceleft tabular\textbraceright" 後面,否則標註會落於表格下方。
    \item 如需定義表格長度,可在後面補上長度指令,例如:\textbackslash makebox[3cm] [l] ,其中 [l] 為置左,也可用置中 [c]或置右 [r] 代替。
    \item {\bf \textcolor{red}{如遭遇表號(caption)打中文而編譯失敗時,可在文件尾端 \textbackslash end \textbraceleft CJK\textbraceright 的前面加入 "\textbackslash clearpage "指令,並刪除.tex檔以外的同名檔案,重新開啟 Latex 即可。}}

\vspace{0.5cm}
    \item 範例:

\end{itemize}

\begin{lstlisting}
    \begin{table}[h]
        \caption{Example 1}
        \label{t1}
        \centering
        \begin{tabular}{|l|c|r|}
        \hline
        \makebox[3cm][l]{Name} & Height(cm) & Weight(kg) \\\hline
        ChangCJ & 175  & 65  \\\hline
        LaiMC & 171  & 55  \\\hline
        YenCY & 177  & 70  \\\hline
        \end{tabular}
    \end{table}
\end{lstlisting}

\vspace{-0.5cm}
    \begin{table}[h]
        \caption{Example 1}
        \label{t1}
        \centering
        \begin{tabular}{|l|c|r|p|}
        \hline
        \makebox[3cm][l]{Name} & Height(cm) & Weight(kg) \\\hline
        ChangCJ & 175  & 65   \\\hline
        LaiMC & 171  & 55  \\\hline
        YenCY & 177  & 70  \\\hline
        \end{tabular}
    \end{table}

\newpage
\section{有斜線的表格}
當需製作斜線的表格,則需先在Latex套件使用區鍵入\textbackslash usepackage\textbraceleft slashbox\textbraceright ,再使用 \textbackslash backslashbox 或 \textbackslash slashbox 指令即可製作表格,範例如下表 1.2 所示。

\vspace{0.5cm}
\begin{itemize}
    \item 範例:
\end{itemize}

\begin{lstlisting}
    \begin{table}[h]
	\begin{center}
	\caption{Example 2}
	\begin{tabular}{|c||c|c||c|}\hline
	\backslashbox{Test}{Property}&\makebox[3cm]{A}&
	\makebox[3cm]{B}&\slashbox{Property}{Number}\\
	\hline \hline
	1 & 2.117 & 12.338 & No.1\\ \hline
	2 & 2.149 & 11.995 & No.2\\ \hline
	\end{tabular}
	\end{center}
    \end{table}
\end{lstlisting}

\vspace{-0.5cm}
\begin{table}[h]
\begin{center}
\caption{Example 2}
\begin{tabular}{|c||c|c||c|}\hline
\backslashbox{Test}{Property}& \makebox[3cm]{A}& 
\makebox[3cm]{B}&\slashbox{Property}{Number} \\
\hline \hline
 1 & 2.117 & 12.338 & No.1\\ \hline
 2 & 2.149 & 11.995 & No.2\\ \hline
\end{tabular}
\end{center}
\end{table}

\section{調整線寬、空間及文字換行}

\subsection{調整線寬 - 列}
不需新增套件,但需在文件開始(\textbackslash begin\textbraceleft document\textbraceright)之前鍵入以下編碼,再把需要改線寬\textbackslash hline 的部分改為 \textbackslash hlinewd \textbraceleft width\textbraceright,如 \textbackslash hlinewd \textbraceleft 5pt\textbraceright 即可。\\

\begin{lstlisting}
	\makeatletter
	\def\hlinewd#1{
	\noalign{\ifnum0=`}\fi\hrule \@height #1
	\futurelet\reserved@a\@xhline}
	\makeatother
\end{lstlisting}

\vspace{-0.5cm}
\begin{table}[h]
\begin{center}
\caption{Example 3}
\begin{tabular}{|c|c|c|} 
\hlinewd{1pt}
1 & 2 & 3\\ \hlinewd{3pt}
4 & 5 & 6\\ \hlinewd{5pt}
\end{tabular}
\end{center}
\end{table}

\subsection{調整線寬 - 行}
行寬並無法像列寬能直接鍵入 \textbackslash hlinewd,需自行定義一個代碼(一般預設行線為 $|$),以在需要更改線寬時使用。若令此代碼為 k,且線寬為 3pt,則在文件開始前,我們需鍵入以下編碼來定義 k。
\vspace{0.5cm}

\begin{lstlisting}
	\newcolumntype{k}{!{\vrule width 3pt}}
\end{lstlisting}

\vspace{0.5cm}
鍵入該編碼後,即可利用此代碼製作表格(使用時機同 $|$ ),如需多種線寬,則可自行多設定代碼以備使用之需。

\vspace{0.5cm}
\begin{itemize}
    \item 範例:
\end{itemize}

\vspace{-0.5cm}
\begin{lstlisting}
	\begin{table}[h]
	\begin{center}
	\caption{Example 4}
	\begin{tabular}{|c k c|c|} \hline
	1 & 2 & 3  \\ \hlinewd{2pt}
	4 & 5 & 6  \\ \hlinewd{3pt}
	\end{tabular}
	\end{center}
	\end{table}
\end{lstlisting}

\vspace{-0.5cm}
\begin{table}[h] 
\begin{center}
\caption{Example 4}
\begin{tabular}{|c k c|c|} \hline
1 & 2 & 3  \\ \hlinewd{2pt}
4 & 5 & 6  \\ \hlinewd{3pt}
\end{tabular}
\end{center}
\end{table}


\subsection{行高和列寬調整}
當需要調整行高或列寬時,可在製作表格指令(\textbackslash begin\textbraceleft table\textbraceright)後面補上對應編碼,即可設置行高(1)、增減列寬(2)或將表格尺寸最小化(3)。表 1.5 為將表 1.4 列寬增加10pt的結果,其編碼則不再重複提及。\\

\begin{lstlisting}
	(1) \begin{table} \renewcommand{\arraystretch}{1.5}   
	(2) \begin{table} \addtolength{\tabcolsep}{10pt}  
	(3) \begin{table} \small   
\end{lstlisting}


\begin{table}[h] \addtolength{\tabcolsep}{10pt}  
\begin{center}
\caption{Example 5}
\begin{tabular}{|c k c|c|} \hline
1 & 2 & 3  \\ \hlinewd{2pt}
4 & 5 & 6  \\ \hlinewd{3pt}
\end{tabular}
\end{center}
\end{table}

\newpage
\subsection{文字換行}
若欲對表格的文字換行,則需自行在文件中定義一個命令:\textbackslash tabincell,其編碼如以下所示。定義完畢後,即可用\textbackslash tabincell\textbraceleft \textbraceright 任意對表格內的文字編排(即大括號可使用 \textbackslash\textbackslash 換行,對齊方式仍然可用 c、l、r)。


\vspace{0.5cm}
\begin{lstlisting}
\newcommand{\tabincell}[2]{\begin{tabular}{@{}#1@{}}#2\end{tabular}}
\end{lstlisting}

\vspace{0.5cm}
\begin{itemize}
    \item 範例:
\end{itemize}

\vspace{-0.5cm}
\begin{lstlisting}
	\begin{table}[h]
	\caption{Example 6}
	\newcommand{\tabincell}[2]{\begin{tabular}{@{}#1@{}}#2\end{tabular}}
 	\centering
 	\begin{tabular}{|c|c|c|}\hline
	1 & \tabincell{c}{Single \\ Double \\ Triple \\ Quadra \\ Penta} 
	  & \tabincell{c}{X}\\\hline
	2 & \tabincell{c}{ S \\ O \\ Lab}
	  & \tabincell{c}{System \\ Optimization \\ Laboratory} \\\hline
	\end{tabular}
	\end{table}
\end{lstlisting}


\vspace{-0.5cm}
\begin{table}[h]
 \caption{Example 6}
\newcommand{\tabincell}[2]{\begin{tabular}{@{}#1@{}}#2\end{tabular}}
  \centering
  \begin{tabular}{|c|c|c|}\hline
1 & \tabincell{c}{Single \\ Double \\ Triple \\ Quadra \\ Penta \\ Hexa } 
   & \tabincell{c}{X}\\\hline
2 & \tabincell{c}{ S \\ O \\ Lab}
   & \tabincell{c}{System \\ Optimization \\ Laboratory} \\\hline
\end{tabular}
\end{table}


\chapter{合併型表格}
此功能類似Office表格的「合併儲存格」,但Latex編譯此表格則需用線條一一繪出表格,其主要指令意義如下所示:
	\begin{enumerate}
	\item \textbackslash multicolumn\textbraceleft 占用欄位數\textbraceright \textbraceleft 對齊方式\textbraceright \textbraceleft 文字內容\textbraceright。
	\item \textbackslash multirow\textbraceleft 占用列位數\textbraceright \textbraceleft 對齊方式\textbraceright \textbraceleft 文字內容\textbraceright。
	\item \textbackslash cline \textbraceleft a-b\textbraceright:如 \textbackslash cline \textbraceleft 2-3\textbraceright 為畫第二欄至第三欄的橫線。
	\end{enumerate}

\vspace{0.5cm}
\begin{itemize}
    \item 範例:
\end{itemize}

\vspace{-0.5cm}

\begin{lstlisting}
	\begin{table}[h]
	\begin{center}
	\caption{Multi Example}
	\begin{tabular}{|c|c|c|c|c|}
	\hline
	\multirow{2}{*}{ABC} &
	\multicolumn{2}{c|}{DEF} &
	\multicolumn{2}{c|}{\multirow{2}{*}{GHI}} \\
	\cline{2-3}
	& 123 & 456 & \multicolumn{2}{c|}{} \\ \hline
	label-1 & label-2 & label-3 & label-4 & label-5 \\ \hline
	\end{tabular}
	\end{center}
	\end{table}
\end{lstlisting}



\begin{table} [h]
\begin{center}
\caption{Multi Example}
\begin{tabular}{|c|c|c|c|c|}
\hline
\multirow{2}{*}{ABC} &
\multicolumn{2}{c|}{DEF} &
\multicolumn{2}{c|}{\multirow{2}{*}{GHI}} \\
\cline{2-3}
  & 123 & 456 & \multicolumn{2}{c|}{} \\
\hline
label-1 & label-2 & label-3 & label-4 & label-5 \\
\hline
\end{tabular}
\end{center}
\end{table}






\chapter{大型表格應用}
\section{跨頁表格}
當表格過長,出現跨頁問題時,普通 Latex 的製表指令已無法使用,需改用長表格 \textbackslash\textbraceleft longtable\textbraceright 代替之,使用之要點如下:
\begin{itemize}
    \item 需先在Latex套件使用區鍵入\textbackslash usepackage\textbraceleft longtable\textbraceright。
    \item 表號指令 "\textbackslash caption"不可寫在 "\textbackslash end\textbraceleft tabular\textbraceright" 後面,否則標註會落於表格下方。

\vspace{0.5cm}
\item 範例:
\end{itemize}


\vspace{-0.5cm}
\begin{lstlisting}
	\begin{center}
	\begin{longtable}{|c|c|c|} 
	\caption{CVT Components} \label{t1}\\ \hline
	\multicolumn{1}{|c|}{Components}&
	\multicolumn{1}{|c|}{Parameters}&
	\multicolumn{1}{|c|}{Physical Meaning} \\ \hline \hline
			.
			.
			.
	\multirow{5}*{Moving Plate}&
	\multicolumn{1}{c|}{$ I_w $}&
	\multicolumn{1}{c|}{''} \\ 
	\cline{2-3} & $ \omega_w $ & '' \\
	\cline{2-3} & $ \alpha $ & '' \\
	\cline{2-3} & $ \beta_w $ & '' \\
	\cline{2-3} & $ \mu_{wb} $ &'' \\ \hline
	\end{longtable}
	\end{center}
\end{lstlisting}

\vspace{-0.5cm}
\begin{center}
\begin{longtable}{|c|c|c|} 
\caption{CVT Components} \label{t1}\\
\hline
\multicolumn{1}{|c|}{Components}&
\multicolumn{1}{|c|}{Parameters}&
\multicolumn{1}{|c|}{Physical Meaning} \\ \hline \hline

\multirow{3}*{Input Shaft}&
\multicolumn{1}{c|}{$ I_i $}&
\multicolumn{1}{c|}{Moment of inertia of input shaft } \\ 
\cline{2-3} & $ \omega_i$ & Angular velocity of input shaft  \\
\cline{2-3} & $ T_i $ & Torque acting on input shaft \\ \hline \hline

\multirow{3}*{Moving Plate}&
\multicolumn{1}{c|}{$ I_w $}&
\multicolumn{1}{c|}{''} \\ 
\cline{2-3} & $ \omega_w $ & '' \\
\cline{2-3} & $ \mu_{wb} $ &'' \\ \hline
\end{longtable}
\end{center}


\section{過寬表格}
當表格寬度過寬,無法放置在一頁時,可選擇兩種方法改善此問題:
	\begin{itemize}
	\item 將表格按比例縮小,此方法需要在文件開始前鍵入 \textbackslash usepackage\textbraceleft graphix\textbraceright,最後使用 \textbackslash resizebox\textbraceleft \textbackslash textwidth\textbraceright \textbraceleft !\textbraceright \textbraceleft 表格程式碼\textbraceright 指令,使表格縮減。但此法可能會造成表格字型和字體,甚至格式的改變問題。

\vspace{0.5cm}
\begin{lstlisting}
	\resizebox{\textwidth}{!}{ ... }
\end{lstlisting}

\vspace{0.5cm}
	\item 範例:
\end{itemize}

\vspace{-0.5cm}
\begin{lstlisting}
	\begin{center}
	\begin{table}[h]
	\resizebox{\textwidth}{!}{ 
	\begin{tabular}{|l|c|c|}
	\hline
	\makebox[3cm][l]{Name} & Height(cm) & Weight(kg) \\\hline
	ChangCJ & 175  & 65 \\\hline
	LaiMC & 171  & 55 \\\hline
	YenCY & 177  & 70 \\\hline
	\end{tabular}}
	\end{table}
	\end{center}
\end{lstlisting}

	\begin{center}
	\begin{table}[h]
	\caption{Sizing Example}
	\resizebox{\textwidth}{!}{ 
	\begin{tabular}{|l|c|c|}
	\hline
	\makebox[3cm][l]{Name} & Height(cm) & Weight(kg) \\\hline
	ChangCJ & 175  & 65 \\\hline
	LaiMC & 171  & 55 \\\hline
	YenCY & 177  & 70 \\\hline
	\end{tabular}}
	\end{table}
	\end{center}


另一方法為:直接旋轉表格90度,使表格能直接放入頁面內。此指令需要用到套件 \textbackslash usepackage\textbraceleft rotating\textbraceright。以上資料截自:http://edt1023.sayya.org/tex/latex123/node9.html

\vspace{0.5cm}
\begin{itemize}
    \item 範例:
\end{itemize}

\vspace{-0.5cm}
\begin{lstlisting}
	\begin{sidewaystable}
	\centering
	\begin{tabular}{llllllllp{1in}lp{1in}}
		.
		.
		.
	\end{tabular}
	\end{sidewaystable}
\end{lstlisting}



\begin{sidewaystable}
\centering
\caption[Grooved Ware and Beaker Features, their Finds and
Radiocarbon Dates]{Grooved Ware and Beaker Features, their
Finds and Radiocarbon Dates; For a breakdown of the Pottery
Assemblages see Tables I and III; for
the Flints see Tables II and IV; for the
Animal Bones see Table V.}
\begin{tabular}{llllllllp{1in}lp{1in}}
\toprule
Context   &Length   &Breadth/   &Depth   &Profile   &Pottery   &Flint   &Animal   &Stone   &Other    &C14 Dates \\
  &         &Diameter   &        &          &          &        & 
Bones&&&\\
\midrule
&&&&&&&&&&\\
\multicolumn{10}{l}{\bf Grooved Ware}&\\
784       &---        &0.9m       &0.18m   &Sloping U &P1       &$\times$46  &  $\times$8      &&       $\times$2 bone&  2150$\pm$ 100 BC\\
785       &---        &1.00m      &0.12    &Sloping U &P2--4    &$\times$23  &  $\times$21     & Hammerstone &---&---\\
962       &---        &1.37m      &0.20m   &Sloping U &P5--6    &$\times$48  &  $\times$57*    & ---&     ---&1990 $\pm$ 80 BC (Layer 4) 1870 $\pm$90 BC (Layer 1)\\
983       &0.83m      &0.73m      &0.25m   &Stepped U &---      &$\times$18  &  $\times$8      & ---& Fired clay&---\\
&&&&&&&&&&\\
\multicolumn{10}{l}{\bf Beaker}&\\
552       &---        &0.68m      &0.12m   &Saucer    &P7--14   &---           & ---       & ---       &---        &---\\
790       &---        &0.60m      &0.25m   &U         &P15      &$\times$12    & ---       & Quartzite-lump&---    &---\\
794       &2.89m      &0.75m      &0.25m   &Irreg.    &P16      &$\times$3     & ---       & ---       &---        &---\\
\bottomrule
\end{tabular}
\end{sidewaystable}

\chapter{未解決問題}

\begin{enumerate}
	\item 有兩條斜線(畫三等份)的表格。
	\item 如何把EXCEL檔案的資料直接轉成Latex表格。
	\item 如何調整頁面,使過寬表格不需經由縮小化或翻轉來強制放入。
\end{enumerate}

\section{備註}
	給接下去製作說明書的同學或學弟妹:\\

此份文件由於「可能」出現無解(中文問題)的編譯BUG,故不建議直接從此份文件接續下去修改,最好另外製作一份文件,在編譯無誤後再貼回此份說明書,以避免兩頭空的問題。



\clearpage
\end{CJK}
\end{document}
